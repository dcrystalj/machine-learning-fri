% To je predloga za poročila o domačih nalogah pri predmetih, katerih
% nosilec je Blaž Zupan. Seveda lahko tudi dodaš kakšen nov, zanimiv
% in uporaben element, ki ga v tej predlogi (še) ni. Več o LaTeX-u izveš na
% spletu, na primer na http://tobi.oetiker.ch/lshort/lshort.pdf.
%
% To predlogo lahko spremeniš v PDF dokument s pomočjo programa
% pdflatex, ki je del standardne instalacije LaTeX programov.

\documentclass[a4paper,11pt]{article}
\usepackage{a4wide}
\usepackage{fullpage}
\usepackage[utf8x]{inputenc}
\usepackage[slovene]{babel}
\selectlanguage{slovene}
\usepackage[toc,page]{appendix}
\usepackage[pdftex]{graphicx} % za slike
\usepackage{setspace}
\usepackage{color}
\definecolor{mygreen}{rgb}{0,0.6,0}
\definecolor{mygray}{rgb}{0.5,0.5,0.5}
\definecolor{mymauve}{rgb}{0.58,0,0.82}

\usepackage{listings}
\lstset{ %
  backgroundcolor=\color{white},   % choose the background color; you must add \usepackage{color} or \usepackage{xcolor}
  basicstyle=\footnotesize,        % the size of the fonts that are used for the code
  breakatwhitespace=false,         % sets if automatic breaks should only happen at whitespace
  breaklines=true,                 % sets automatic line breaking
  captionpos=b,                    % sets the caption-position to bottom
  commentstyle=\color{mygreen},    % comment style
  deletekeywords={...},            % if you want to delete keywords from the given language
  escapeinside={\%*}{*)},          % if you want to add LaTeX within your code
  extendedchars=true,              % lets you use non-ASCII characters; for 8-bits encodings only, does not work with UTF-8
  frame=single,                    % adds a frame around the code
  keepspaces=true,                 % keeps spaces in text, useful for keeping indentation of code (possibly needs columns=flexible)
  keywordstyle=\color{blue},       % keyword style
  language=Octave,                 % the language of the code
  otherkeywords={*,...},            % if you want to add more keywords to the set
  numbers=left,                    % where to put the line-numbers; possible values are (none, left, right)
  numbersep=5pt,                   % how far the line-numbers are from the code
  numberstyle=\tiny\color{mygray}, % the style that is used for the line-numbers
  rulecolor=\color{black},         % if not set, the frame-color may be changed on line-breaks within not-black text (e.g. comments (green here))
  showspaces=false,                % show spaces everywhere adding particular underscores; it overrides 'showstringspaces'
  showstringspaces=false,          % underline spaces within strings only
  showtabs=false,                  % show tabs within strings adding particular underscores
  stepnumber=2,                    % the step between two line-numbers. If it's 1, each line will be numbered
  stringstyle=\color{mymauve},     % string literal style
  tabsize=2,                       % sets default tabsize to 2 spaces
  title=\lstname                   % show the filename of files included with \lstinputlisting; also try caption instead of title
}

\definecolor{light-gray}{gray}{0.95}

\usepackage{listings} % za vključevanje kode
\usepackage{hyperref}
\renewcommand{\baselinestretch}{1.2} % za boljšo berljivost večji razmak
\renewcommand{\appendixpagename}{Priloge}

\lstset{ % nastavitve za izpis kode, sem lahko tudi kaj dodaš/spremeniš
language=Python,
basicstyle=\footnotesize,
basicstyle=\ttfamily\footnotesize\setstretch{1},
backgroundcolor=\color{light-gray},
}

\title{Gradientni boosting}
\author{Tomazic Tomaz (63100281)}
\date{\today}

\begin{document}

\maketitle

\section{Uvod}

Cilj naloge je bil napovedati eno izmed devetih skupin, kateri pripada nek izdelek.

\section{Podatki}

Podatki so imeli 93 značilk katere so bile anonime. Vsi podatki so zasegali diskretne vrednosti in so bili nenegativni.
Na voljo je bilo 50000 učnih primerov, v katerih sta močno prevladala drugi in šesti razred.

\section{Metode}

Implementiral sem gradientni boosting ki za funkcijo izgube uporablja KL-divergenco (Kullback Leibler divergence). Razred GradBoost ob inicializaciji sprejme kot parameter šibak regresijski model, ki mora imeti implementirani funkciji `predict` in `fit`. Kot parameter sprejme tudi ime funkcije izgube. Trenutno je implementirana samo KL izguba. Razred `GradBoost` deluje za vecrazredno klasifikacijo.

V fit metodi kot osnovno ucenje najprej izracunam frekvenco razredov nato pa izboljsujem napoved z regresijskim drevesom oz podanim ucnim modelom.

Napovedne modele sem ovrednotil z precnim preverjanjem v katerem sem iskal najboljse parametre za dane podatke. Ti parametri so: stevilo modelov za posamezni razred, globina regresijskega drevesa in regularizacijski parameter `ro`. V precnem preverjanju sem poizkusal 30 modelov med 10 in 1000, globine drevesa od 1 do 5 in 5 regularizacijskij prametrov med 0.01 in 1.

Za zdruzevanje napovedi nisem uporabil metode stacking ampak samo povprecenje rezultatov mojih nevronskih mrez in rezultate gradient boostinga, ker sem vseeno pricakoval nekoliko boljse rezultate na lestvici.

\section{Rezultati}

Presenetilo me je da sem najboljse rezultate dobil z regularizacijskim parametrom vrednosti 1.

V tabeli~\ref{tab1} so najboljsi rezultati pri izbiri paramerov za doloceno globino drevesa. V tabeli-\ref{tab2} pa rezultati precnega preverjanja
\begin{table}[htbp]
\caption{Rezultati povprecja treh foldov pri razlicnih parametrih.}
\label{tab1}
\begin{center}
\begin{tabular}{cccl}
\hline
st iteracij & globina drevesa & ro & rezultat \\
\hline
500 & 1 & 1 & 0.66513236171212353 \\
500 & 2 & 1 & 0.57993880037618517 \\
500 & 3 & 1 & 0.54497348832603998 \\
500 & 4 & 1 & 0.53358386235816502 \\
319 & 5 & 1 & 0.53697762802747351 \\
625 & 4 & 1 & 0.53264532029139089 \\
\hline
\end{tabular}
\end{center}
\end{table}

\begin{table}[htbp]
\caption{Rezultati precnega preverjanja.}
\label{tab2}
\begin{center}
\begin{tabular}{lccccc}
\hline
ime & st iteracij & globina drevesa & ro & vrednost modela & oddaja na strezniku (9b7) \\
\hline
boosting & 625 & 4 & 1 & 0.51243599428 & 0.51648 \\
\hline
nevronske mreze &   & 51 & 0.003 & 0.54092656750 & 0.53926 \\
povprecenje &  &  &  &  & 0.49849 \\
\hline
\end{tabular}
\end{center}
\end{table}

Rezultati ocitno kazejo da je boosting boljsa metoda v primerjavi z nevronskimi mrezami za dane podatke.

\section{Izjava o izdelavi domače naloge}
Domačo nalogo in pripadajoče programe sem izdelal sam.

\appendix
\appendixpage

\section{\label{app-code}Programska koda}


\begin{lstlisting}
import numpy as np
import sklearn
import copy
from Frequency import Frequency

class GradBoost:
    """Gradient Boosting for Classification."""

    def __init__(self, learner, n_estimators=100, loss="KL", epsilon=1e-5):
        self.n_estimators = n_estimators
        self.learner      = learner
        self.epsilon      = epsilon

        losses = {
            "huber": self.grad_huber_loss,
            "squared": self.grad_squared_loss,
            "abs": self.grad_abs_loss,
            "KL": self.kl_loss
        }
        self.loss = losses[loss]

    def grad_squared_loss(self, y, f):
        """Negative gradiant for squared loss."""
        return y - f

    def grad_abs_loss(self, y, f):
        """Negative gradient for absolute loss."""
        return np.sign(y - f)

    def grad_huber_loss(self, y, f, delta=0.5):
        """Negative gradient for Huber loss."""
        r0 = y - f
        r1 = delta * np.sign(r0)
        return np.vstack((r0, r1)).T[np.arange(y.shape[0]), (np.abs(r0)>delta).astype(int)]

    def kl_loss(self, y, f):
        """Negative gradient for kullback leibler."""
        return y - f

    def fit(self, X, Y, ro=1):
        num_c = Y.shape[1]

        #average
        models = [[Frequency().fit(X, Y[:, i]) for i in range(num_c)]]
        F      = np.hstack([i.predict(X) for i in models[0]])

        grad = self.loss(Y, F)

        for i in range(self.n_estimators):

            models.append([])

            for j in range(num_c):
                #fit regression tree
                dtc = copy.copy(self.learner)
                dtc.fit(X, grad[:, j])
                models[i+1].append(dtc)

                F[:, j] += dtc.predict(X) * ro

            P    = self.softmax(F)
            grad = self.loss(Y, P)

        self.models = np.array(models)
        return self


    def softmax(self, f):
        s = np.exp(f - np.max(f, axis=1)[:,None])
        s /= np.sum(s, axis=1)[:, None]
        return s

    def test_grad(self, Y):
        F = np.zeros(Y.shape)
        P = self.softmax(F)
        grad = self.loss(Y, P)

        # print(self.init_thetas)
        approx = np.absolute(self.grad_approx(Y, P))
        real = np.absolute(self.loss(Y, P))
        print( np.absolute(sum((approx - real))) )
        print( np.sum((approx - real)**2) )

    def KL(self, Q, P):
        return np.sum(P * np.log(P / (np.exp(Q)/np.sum(Q))))

    def grad_approx(self, Y, P, e=1e-4):
        num_grad = np.zeros_like(Y)
        perturb = np.zeros(Y.shape[0])
        for j in range(Y.shape[1]):
            for i in range(Y.shape[0]):
                perturb[i] = e
                j1 = self.KL(Y[:,j], P[:,j] + perturb)
                j2 = self.KL(Y[:,j], P[:,j] - perturb)
                num_grad[i, j] = (j1 - j2) / (2. * e)
                perturb[i] = 0
        return self.softmax(num_grad)

    def predict_proba(self, X):
        return self.predict(X)

    def predict(self, X):
        models = self.models
        num_c  = models.shape[1]

        P = np.hstack([i.predict(X) for i in models[0]])

        for i in range(num_c):
            P[:, i] += np.sum([j.predict(X) for j in models[1:, i]], axis=0)

        return self.softmax(P)

    def get_params(self, deep=False):
        return {
            "learner": self.learner,
            "n_estimators": self.n_estimators
        }


    ##################### Frequency.py ####################
    import numpy as np

    class Frequency:
        def fit(self, X, y):
            self.model = np.sum(y, axis=0) / y.shape[0]
            return self

        def predict(self, X):
            return np.vstack([self.model for i in range(X.shape[0])])

    ##################### CV.py ####################

    from GradBoost import GradBoost
    import IO
    import numpy as np
    import Orange
    from sklearn import preprocessing, metrics, grid_search
    from sklearn.tree import DecisionTreeRegressor
    from sklearn.cross_validation import cross_val_score, ShuffleSplit, train_test_split


    def cv_score(X, Y, split, n_estimators, depth):
        learner = DecisionTreeRegressor(max_depth=depth)
        gb = GradBoost(learner, n_estimators)
        return np.absolute(np.mean(cross_val_score(gb, X, Y, \
            cv=split, scoring='log_loss', n_jobs=4)
        ))

    def find_params(X, Y): #n_estimators && depth
        n_estimators = list(map(int, np.linspace(10, 500, 20)))
        depths = list(map(int, np.linspace(1, 5, 5)))

        cv_split = ShuffleSplit(Y.shape[0], n_iter=3, test_size=0.3, random_state=42)

        scores = []
        for j in depths:
            scores.append( min([(cv_score(X, Y, cv_split, i, j), i, j) for i in n_estimators]) )
            print(scores[-1])
        print(scores)
        score = min(scores)
        print("best n_estimators", score[1], "depth", score[2], " got mean score ", score[0], " for 3 folds")
        return (score[1], score[2])

    def eval_model(X, Y):
        '''evaluate model with best lambda on unseen data'''

        X_train, X_test, Y_train, Y_test = train_test_split(
            X, Y, test_size=0.2, random_state=42
        )

        n_estimators, depth = find_params(X_train, Y_train)
        learner = DecisionTreeRegressor(max_depth=depth)
        gb = GradBoost(learner, n_estimators)
        gb.fit(X_train, Y_train)
        y = gb.predict(X_test)
        result = metrics.log_loss(Y_test, y)
        print("this model got score ", result, " with n ", n_estimators)
        return (n_estimators, depth)

    def predict(X_train, Y_train, X, filename="result"):
        n_estimators, depth = eval_model(X_train, Y_train)

        learner = DecisionTreeRegressor(max_depth=depth)
        gb = GradBoost(learner, n_estimators)
        gb.fit(X_train, Y_train)
        prediction = gb.predict(X)
        IO.savePrediction(prediction, filename)
        print("prediction finished")

    def simple_prediction_test(X, Y, lambda_=0):
        learner = DecisionTreeRegressor(max_depth=4)
        gb = GradBoost(learner)

        X_train, X_test, Y_train, Y_test = train_test_split(
            X, Y, test_size=0.2, random_state=42
        )
        gb.fit(X_train, Y_train)
        prediction = gb.predict(X_test)
        # print(prediction)
        print(metrics.log_loss(Y_test, prediction))

    ##################### IO.py ####################

    import csv
    import numpy as np
    from sklearn import preprocessing

    def readFile(path="data4_reduced.csv"):
        with open(path, newline='') as csvfile:
            data = csv.reader(csvfile, delimiter=',')
            data = [i[1:-1] + [i[-1][-1]] for i in data]

        data.pop(0)
        data = np.array(data).astype(int)
        dataX = data[:, :-1]
        dataY = data[:, -1] #from 1 to 9
        dataY = np.eye(np.max(dataY))[dataY.astype(int)-1]
        return dataX, dataY

    def readTestFile(path="test.csv"):
        with open(path, newline='') as csvfile:
            data = csv.reader(csvfile, delimiter=',')
            data = [i[1:] for i in data]

        data.pop(0)
        data = np.array(data).astype(int)
        return data

    def savePrediction(p, name="result"):
        f = open(name + ".csv", 'w')
        f.write("id,Class_1,Class_2,Class_3,Class_4,Class_5,Class_6,Class_7,Class_8,Class_9\n")
        np.set_printoptions(suppress=True)
        np.set_printoptions(precision=7)
        for i, j in enumerate(p):
            f.write(str(i+1) + "," + ",".join(j.astype(str)) + "\n")
        f.close()

    def normalize(X, X_test=None):
        p = preprocessing.Normalizer().fit(X)
        X = p.transform(X)
        if X_test is None:
            return (X)
        else:
            X_test = p.transform(X_test)
            return (X, X_test)



    ##################### boost.py ####################

    import Orange
    import numpy as np
    import sklearn
    import CV
    import IO
    from GradBoost import GradBoost
    from sklearn.tree import DecisionTreeRegressor as DTC

    numClasses = 3
    iris = Orange.data.Table("iris")
    X = iris.X
    Y = np.eye(numClasses)[iris.Y.astype(int)]

    # print(GradBoost().fit(X, y))
    # print(CV.simple_prediction_test(X, Y))
    # print(CV.eval_model(X, Y))


    # X, Y = IO.readFile()
    # print(CV.simple_prediction_test(X, Y))
    # print(CV.eval_model(X, Y))

    X, Y = IO.readFile("train.csv")
    Y_test = IO.readTestFile()
    CV.predict(X, Y, Y_test, "result")

\end{lstlisting}

\end{document}
