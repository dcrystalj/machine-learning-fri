% To je predloga za poročila o domačih nalogah pri predmetih, katerih
% nosilec je Blaž Zupan. Seveda lahko tudi dodaš kakšen nov, zanimiv
% in uporaben element, ki ga v tej predlogi (še) ni. Več o LaTeX-u izveš na
% spletu, na primer na http://tobi.oetiker.ch/lshort/lshort.pdf.
%
% To predlogo lahko spremeniš v PDF dokument s pomočjo programa
% pdflatex, ki je del standardne instalacije LaTeX programov.

\documentclass[a4paper,11pt]{article}
\usepackage{a4wide}
\usepackage{fullpage}
\usepackage[utf8x]{inputenc}
\usepackage[slovene]{babel}
\selectlanguage{slovene}
\usepackage[toc,page]{appendix}
\usepackage[pdftex]{graphicx} % za slike
\usepackage{setspace}
\usepackage{color}
\definecolor{light-gray}{gray}{0.95}
\definecolor{mygreen}{rgb}{0,0.6,0}
\definecolor{mygray}{rgb}{0.5,0.5,0.5}
\definecolor{mymauve}{rgb}{0.58,0,0.82}


\usepackage{listings}
\lstset{ %
  backgroundcolor=\color{white},   % choose the background color; you must add \usepackage{color} or \usepackage{xcolor}
  basicstyle=\footnotesize,        % the size of the fonts that are used for the code
  breakatwhitespace=false,         % sets if automatic breaks should only happen at whitespace
  breaklines=true,                 % sets automatic line breaking
  captionpos=b,                    % sets the caption-position to bottom
  commentstyle=\color{mygreen},    % comment style
  deletekeywords={...},            % if you want to delete keywords from the given language
  escapeinside={\%*}{*)},          % if you want to add LaTeX within your code
  extendedchars=true,              % lets you use non-ASCII characters; for 8-bits encodings only, does not work with UTF-8
  frame=single,                    % adds a frame around the code
  keepspaces=true,                 % keeps spaces in text, useful for keeping indentation of code (possibly needs columns=flexible)
  keywordstyle=\color{blue},       % keyword style
  language=Octave,                 % the language of the code
  otherkeywords={*,...},            % if you want to add more keywords to the set
  numbers=left,                    % where to put the line-numbers; possible values are (none, left, right)
  numbersep=5pt,                   % how far the line-numbers are from the code
  numberstyle=\tiny\color{mygray}, % the style that is used for the line-numbers
  rulecolor=\color{black},         % if not set, the frame-color may be changed on line-breaks within not-black text (e.g. comments (green here))
  showspaces=false,                % show spaces everywhere adding particular underscores; it overrides 'showstringspaces'
  showstringspaces=false,          % underline spaces within strings only
  showtabs=false,                  % show tabs within strings adding particular underscores
  stepnumber=2,                    % the step between two line-numbers. If it's 1, each line will be numbered
  stringstyle=\color{mymauve},     % string literal style
  tabsize=2,                       % sets default tabsize to 2 spaces
  title=\lstname                   % show the filename of files included with \lstinputlisting; also try caption instead of title
}

\definecolor{light-gray}{gray}{0.95}


\usepackage{listings} % za vključevanje kode
\usepackage{hyperref}
\renewcommand{\baselinestretch}{1.2} % za boljšo berljivost večji razmak
\renewcommand{\appendixpagename}{Priloge}

\lstset{ % nastavitve za izpis kode, sem lahko tudi kaj dodaš/spremeniš
language=Python,
basicstyle=\footnotesize,
basicstyle=\ttfamily\footnotesize\setstretch{1},
backgroundcolor=\color{light-gray},
}

\title{Klasifikacija besedil z jedrnimi metodami}
\author{Tomaž Tomažič (63100281)}
\date{\today}

\begin{document}

\maketitle

\section{Uvod}

V nalogi sem uporabili SVM za klasifikacijo besedil da bi pri tem spoznal pomen jeder.

\section{Podatki}

Podatke smo si morali sami poiskati v treh kategorijah. Shranil sem bloge. Izbral sem si dve podobni kategoriji Windows blog (A) in Bing blog (B). Za drugacno temo pa sem izbral Zoella blog (C), ki je popularen blog za zenske, licenje in podobne reci.

\section{Metode}

Potrebno je bilo napisati jedro za podporne vektorje implementirane v sklearn knjižnjici. Jedro deluje na principu merjenja razdalje s kompresijo, kot je bilo podano v navodilih naloge. Razdalje sem obrnil tako da sem odstel vse vrednosti od najvecje v matriki razdalj. Napovedi sem naredil za vse tri kombinacije med kategorijami: A in B, A in C, B in C. Napoved sem naredil tako, da sem se naucil model na 80\% podatkih, ostalih 20\% pa sem uporabil za napoved verjetnosti razreda.

\section{Rezultati}
Rezultate sem preveril kar z metodo ostrega očesa, ko sem izpisal napovedane vrjetnosti.
Rezultati so bili nekoliko boljsi ko sem napovedoval cisto razlicne razrede (A-C in B-C) v primerjavi z podobnimi razredi (A-B). Menim da bi bilo to mozno se precej izboljsati z iskanjem pravih parametrov.

\section{Izjava o izdelavi domače naloge}
Domačo nalogo in pripadajoče programe sem izdelal sam.

\appendix
\appendixpage

\section{\label{app-code}Programska koda}

\begin{lstlisting}
import zlib
import copy
import numpy as np
from sklearn.svm import SVC
from sklearn.metrics import roc_auc_score, accuracy_score, auc

class SVM:
    def __init__(self, data):
        self.learner = SVC(C = 1.1, kernel=self.string_kernel, probability=True)
        self.data = data

    def Z(self, a):
        return len(zlib.compress(a.encode()))

    def distance(self, a, b):
        Z = self.Z
        a = self.data[int(a)]
        b = self.data[int(b)]

        ab = a+b
        ba = b+a
        return (Z(ab) - Z(a)) / Z(a) + (Z(ba) - Z(b)) / Z(b)

    def string_kernel(self, X, Y):
        M = np.zeros((len(X), len(Y)))
        for i, a in enumerate(X):
            for j, b in enumerate(Y):
                M[i, j] = self.distance(a[0], b[0])
        return M.max() - M

    def fit(self, X, Y):
        return self.learner.fit(X, Y)

    def predict(self, X):
        return self.learner.predict_proba(X)

data_a = [ open("source/windows/"+str(i)+".txt", "rt").read() for i in range(1, 21)]
data_b = [ open("source/bing/"+str(i)+".txt", "rt").read() for i in range(1, 21)]
data_c = [ open("source/zoella/"+str(i)+".txt", "rt").read() for i in range(1, 21)]

def test(data_a, data_b):
    da = data_a[:-4]
    db = data_b[:-4]
    data_ab = copy.deepcopy(da)
    data_ab.extend(db)

    Xab = np.vstack(np.r_[0:len(data_ab)])
    Yab = np.array([0]*len(da) + [1]*len(db))

    s = SVM(data_ab)
    s.fit(Xab, Yab)
    s.data.extend(data_a[-4:])
    s.data.extend(data_b[-4:])

    return s.predict(np.vstack(np.r_[len(da)*2:len(data_a)*2]))

print(test(data_a, data_b))
print(test(data_a, data_c))
print(test(data_b, data_c))
\end{lstlisting}

\end{document}
